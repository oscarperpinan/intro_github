% Created 2019-02-07 jue 18:27
% Intended LaTeX compiler: pdflatex
\documentclass[xcolor={usenames,svgnames,dvipsnames}]{beamer}
\usepackage[utf8]{inputenc}
\usepackage[T1]{fontenc}
\usepackage{graphicx}
\usepackage{grffile}
\usepackage{longtable}
\usepackage{wrapfig}
\usepackage{rotating}
\usepackage[normalem]{ulem}
\usepackage{amsmath}
\usepackage{textcomp}
\usepackage{amssymb}
\usepackage{capt-of}
\usepackage{hyperref}
\usepackage{color}
\usepackage{listings}
\usepackage[spanish]{babel}
\usecolortheme{rose}
\setbeamercolor{alerted text}{fg=Blue}
\setbeamerfont{alerted text}{series=\bfseries}
\setbeamerfont{block title}{series=\bfseries}
\setbeamercolor{block title}{bg=structure.fg!20!bg!50!bg}
\setbeamercolor{block body}{use=block title,bg=block title.bg}
\setbeamertemplate{navigation symbols}{}
\AtBeginSection[]{\begin{frame}[plain]\tableofcontents[currentsection,sectionstyle=show/shaded, subsectionstyle=show/hide]\end{frame}}
\AtBeginSubsection[]{\begin{frame}[plain]\tableofcontents[currentsubsection,sectionstyle=show/shaded,subsectionstyle=show/shaded/hide]\end{frame}}
\lstset{keywordstyle=\color{blue}, commentstyle=\color{gray!90}, basicstyle=\ttfamily\small, columns=fullflexible, breaklines=true,linewidth=\textwidth, backgroundcolor=\color{gray!23}, basewidth={0.5em,0.4em}, literate={¡}{{\textexclamdown}}1 {á}{{\'a}}1 {ñ}{{\~n}}1 {é}{{\'e}}1 {ó}{{\'o}}1 {í}{{\'i}}1 {ú}{{\'u}}1 {º}{{\textordmasculine}}1, showstringspaces=false}
\usepackage{mathpazo}
\hypersetup{colorlinks=true, linkcolor=Blue, urlcolor=Blue}
\usepackage{fancyvrb}
\DefineVerbatimEnvironment{verbatim}{Verbatim}{fontsize=\tiny, formatcom = {\color{black!70}}}
\beamertemplatenavigationsymbolsempty
\setbeamertemplate{footline}[frame number]
\usetheme{Goettingen}
\usefonttheme{serif}
\author{Oscar Perpiñán Lamigueiro}
\date{}
\title{Introducción al control de versiones y trabajo colaborativo con GitHub}
\hypersetup{
 pdfauthor={Oscar Perpiñán Lamigueiro},
 pdftitle={Introducción al control de versiones y trabajo colaborativo con GitHub},
 pdfkeywords={},
 pdfsubject={},
 pdfcreator={Emacs 26.1 (Org mode 9.2)}, 
 pdflang={Spanish}}
\begin{document}

\maketitle

\section{Conceptos básicos}
\label{sec:orgec69093}
\subsection{¿Qué es el control de versiones?}
\label{sec:orge065df9}

\begin{frame}[label={sec:orgf60f00f},plain]{}
\begin{center}
\includegraphics[width=0.9\paperwidth]{figs/phdcomic_finaldoc_1.png}
\end{center}

\url{http://phdcomics.com/comics/archive.php?comicid=1531}
\end{frame}

\begin{frame}[label={sec:org8a51442},plain]{}
\begin{center}
\includegraphics[width=0.9\paperwidth]{figs/phdcomic_finaldoc_2.png}
\end{center}

\url{http://phdcomics.com/comics/archive.php?comicid=1531}
\end{frame}

\begin{frame}[label={sec:org2671a5b},plain]{}
\begin{center}
\includegraphics[width=0.9\paperwidth]{figs/phdcomic_finaldoc_3.png}
\end{center}

\url{http://phdcomics.com/comics/archive.php?comicid=1531}
\end{frame}

\begin{frame}[label={sec:orge250d56}]{¿Qué es el control de versiones y por qué debería importarte?}
\begin{quote}
El control de versiones es un sistema que \alert{registra los cambios}
realizados sobre un archivo o conjunto de archivos a lo largo del
tiempo, de modo que se puedan \alert{recuperar} versiones específicas más
adelante.\footnote{\url{https://git-scm.com/book/es/v1/Empezando-Acerca-del-control-de-versiones}}
\end{quote}
\end{frame}

\begin{frame}[label={sec:org70e5fc8}]{¿Qué es el control de versiones y por qué debería importarte?}
\begin{quote}
El control de versiones es el cuaderno de laboratorio en el
mundo digital. Es lo que los profesionales usan para realizar un
\alert{seguimiento} de lo que han hecho y para \alert{colaborar} con otras
personas. Cada gran proyecto de desarrollo de software se basa en
ello, y la mayoría de los programadores lo utilizan para sus
trabajos. Y \alert{no sirve sólo para software}: libros, documentos, pequeños
conjuntos de datos y cualquier cosa que cambie con el tiempo o que
deba compartirse puede y debe almacenarse en un sistema de control de
versiones.\footnote{\url{https://swcarpentry.github.io/git-novice/}}
\end{quote}
\end{frame}

\begin{frame}[label={sec:org0d01efa}]{Viajar en el tiempo}
\begin{itemize}
\item Nada que haya sido sometido a un control de versiones se pierde jamás (\emph{salvo que realmente quieras eliminarlo\ldots{}})
\item \alert{Todas} las versiones antiguas de un fichero se almacenan: un fichero se puede revertir a un estado anterior sin límites.
\end{itemize}
\end{frame}
\begin{frame}[label={sec:org6d276f4}]{¿Qué? ¿Cuándo? ¿Quién?}
Un sistema de control de versiones registra:
\begin{itemize}
\item El detalle de los cambios realizados.
\item La fecha y hora en la que fueron realizados.
\item La persona que los realizó.
\end{itemize}
\end{frame}

\begin{frame}[label={sec:org7683021}]{Trabajo Colaborativo}
\begin{itemize}
\item Cuando un equipo de personas trabaja conjuntamente en un proyecto, es posible que se produzcan cambios incompatibles en un mismo fichero.
\item El sistema de control de versiones \alert{impide} cambios simultáneos en un fichero. A cambio, permite la \alert{resolución de conflictos} y los documenta.
\end{itemize}
\end{frame}

\subsection{¿Qué son Git y GitHub?}
\label{sec:orgdf3e6f5}

\begin{frame}[label={sec:org406be5d},fragile]{Git es un Sistema de Control de Versiones}
 Git es una herramienta software (accesible mediante línea de comandos con \texttt{git}) que implementa un Sistema de Control de Versiones.

\begin{center}
\includegraphics[width=.9\linewidth]{figs/git_model.png}
\end{center}
\end{frame}

\begin{frame}[label={sec:org2e38673}]{Git es un Sistema de Control de Versiones}
Cada vez que se ejecuta un cambio en una estructura de ficheros controlada con Git, realiza una \guillemotleft{}foto\guillemotright{} del estado de los archivos en ese momento, y guarda una referencia a esa instantánea. 
\begin{center}
\includegraphics[width=.9\linewidth]{figs/git_model.png}
\end{center}
\end{frame}

\begin{frame}[label={sec:org6309702}]{Git es un Sistema de Control de Versiones}
Por eficiencia, Git no almacena los archivos sin modificaciones sino un enlace al archivo anterior idéntico que ya está almacenado

\begin{center}
\includegraphics[width=.9\linewidth]{figs/git_model.png}
\end{center}
\end{frame}

\begin{frame}[label={sec:org91c4e12}]{Los estados de Git}
\begin{itemize}
\item El desarrollador incorpora uno o varios ficheros al control de versiones. (\emph{tracked})
\item Realiza modificaciones en los ficheros (\emph{modified}).
\item Incorpora esos ficheros modificados al área de preparación (\emph{staged}).
\item Finalmente, confirma todos los cambios del área de preparación: se realiza la instantánea de los ficheros. (\emph{committed})
\end{itemize}
\begin{center}
\includegraphics[height=0.4\textheight]{figs/git_estados.png}
\end{center}
\end{frame}

\begin{frame}[label={sec:org99e54ef},fragile]{¿Qué es GitHub?}
 \begin{itemize}
\item GitHub es la plataforma de alojamiento de código más importante a nivel mundial.
\item Emplea el sistema de control de versiones \texttt{git}
\item Ofrece una amplia variedad de funcionalidades
\begin{itemize}
\item Alojamiento de código
\item Revisión de código
\item Trabajo colaborativo
\item Publicación de páginas web
\end{itemize}
\end{itemize}
\end{frame}

\section{Primeros pasos}
\label{sec:org59d51ec}

\begin{frame}[label={sec:org846a4f5}]{Creación de una cuenta en GitHub}
\url{https://github.com/join}

\begin{center}
\includegraphics[width=.9\linewidth]{figs/GitHub_Join.png}
\end{center}

Más información en \href{https://help.github.com/articles/signing-up-for-a-new-github-account/}{New GitHub account}
\end{frame}

\begin{frame}[label={sec:org9e0e81c}]{Instalación de GitHub Desktop}
\url{https://desktop.github.com/}

\begin{center}
\includegraphics[width=.9\linewidth]{figs/GitHub_Desktop.png}
\end{center}
\end{frame}

\begin{frame}[label={sec:org2df645f}]{Conectamos Git, GitHub y GitHub Desktop}
\begin{itemize}
\item Una vez instalado comienza el proceso de autenticación, usando las credenciales del paso anterior\footnote{Más información en \href{https://help.github.com/desktop/guides/getting-started-with-github-desktop/authenticating-to-github/}{Authenticating to GitHub}.}.
\end{itemize}

\begin{center}
\boxed{File > Options > Accounts > Sign\ In}
\end{center}


\begin{itemize}
\item A continuación, conectamos la información de usuario con Git\footnote{Más información en \href{https://help.github.com/desktop/guides/getting-started-with-github-desktop/configuring-git-for-github-desktop/}{Configuring Git}.}.
\end{itemize}

\begin{center}
\boxed{File > Options > Git}
\end{center}
\end{frame}


\begin{frame}[label={sec:orgda60fe2}]{Nuevo repositorio desde github.com}
\url{https://github.com/new}

\begin{center}
\includegraphics[width=.9\linewidth]{figs/GitHub_New_Repository.png}
\end{center}
\end{frame}

\begin{frame}[label={sec:org6d2c842}]{Nuevo repositorio desde GitHub Desktop}
\begin{center}
\boxed{File > New\ Repository}
\end{center}

\begin{center}
\includegraphics[height=0.8\textheight]{figs/Desktop_NewRepository.png}
\end{center}
\end{frame}

\begin{frame}[label={sec:org81fd4c2}]{Clonar un repositorio remoto}
Si hemos creado el repositorio desde github.com (\emph{repositorio remoto}), hay que clonarlo (\emph{copia local}).

\begin{center}
\boxed{File > Clone\ Repository}
\end{center}

\begin{center}
\includegraphics[height=0.7\textheight]{figs/Desktop_CloneRepository.png}
\end{center}
\end{frame}

\begin{frame}[label={sec:orgd742b2e}]{Publicar un repositorio local}
Si hemos creado el repositorio desde GitHub Desktop (\emph{repositorio local}), hay que publicarlo en github.com (\emph{remoto})

\begin{center}
\includegraphics[width=.9\linewidth]{figs/Desktop_PublishRepository.png}
\end{center}
\end{frame}

\begin{frame}[label={sec:orgbc5f7cb},fragile]{Consejos básicos}
 \begin{itemize}
\item Elige bien el \texttt{.gitignore} (adecuado al proyecto). Veáse \url{https://github.com/github/gitignore}.
\item No olvides cumplimentar el \texttt{README.md}. Para el formato veáse \href{https://help.github.com/articles/basic-writing-and-formatting-syntax/}{Formatting syntax}.
\item Elige una licencia adecuada a tu proyecto y a tus intereses actuales y futuros. Veáse \url{https://choosealicense.com}.
\end{itemize}
\end{frame}

\section{Flujo de trabajo con \texttt{git} y \texttt{GitHub}}
\label{sec:org5d257ec}
\subsection{Realizar y confirmar cambios (\texttt{add} y \texttt{commit})}
\label{sec:orgc910cad}
\subsection{Publicar cambios (\texttt{push})}
\label{sec:org438d4c9}
\subsection{Recibir cambios de un repositorio remoto y combinar con una copia local (\texttt{fetch}, \texttt{merge} y \texttt{pull})}
\label{sec:orgaf1a0a2}
\section{Trabajo en colaboración}
\label{sec:org7610f70}

\subsection{Ramas}
\label{sec:org3bd1a2c}

\begin{frame}[label={sec:org85fbe83},fragile]{Rama \texttt{master}}
 \begin{center}
\includegraphics[width=.9\linewidth]{figs/branching.png}
\end{center}

En un repositorio de GitHub existe una rama (\emph{branch}) que se usa por defecto: \alert{master}.
\end{frame}

\begin{frame}[label={sec:org91cf1df}]{Ramas para facilitar la colaboración}
\begin{center}
\includegraphics[width=.9\linewidth]{figs/branching.png}
\end{center}

\begin{columns}
\begin{column}{0.6\columnwidth}
Cuando hay varias personas trabajando sobre un mismo repositorio, es necesario crear nuevas ramas para evitar conflictos. De esta forma, cada persona hace cambios en el código en una rama específica.
\end{column}

\begin{column}{0.4\columnwidth}
\begin{center}
\includegraphics[width=.9\linewidth]{figs/nueva_rama_desktop.png}
\end{center}
\end{column}
\end{columns}
\end{frame}

\begin{frame}[label={sec:org03cdb49}]{Combinación de código}
\begin{center}
\includegraphics[width=.9\linewidth]{figs/branching.png}
\end{center}

Cuando los cambios están listos y confirmados (\emph{commit} + \emph{push} en la rama específica), se realiza una petición (\emph{pull request}) para combinar estos cambios en la rama \alert{master}.

\begin{columns}
\begin{column}{0.45\columnwidth}
\begin{center}
\includegraphics[width=.9\linewidth]{figs/pull_request_desktop.png}
\end{center}
\end{column}

\begin{column}{0.45\columnwidth}
\begin{center}
\includegraphics[width=.9\linewidth]{figs/pull_request_web.png}
\end{center}
\end{column}
\end{columns}
\end{frame}

\begin{frame}[label={sec:orgc6cf166}]{Combinación de código}
\begin{center}
\includegraphics[width=.9\linewidth]{figs/branching.png}
\end{center}


El coordinador del proyecto es el encargado de revisar cada petición y, si todo está correcto, incluir los cambios (\emph{merge}) en la rama \alert{master}. 

\begin{center}
\includegraphics[width=.9\linewidth]{figs/merge_pull_request.png}
\end{center}
\end{frame}

\begin{frame}[label={sec:orgf54e116}]{Resolución de conflictos}
Si no se pueden combinar los cambios automáticamente se produce un conflicto (por ejemplo, cuando dos usuarios modifican un mismo fichero).

\begin{center}
\includegraphics[width=.9\linewidth]{figs/conflict_web.png}
\end{center}

Un conflicto se debe resolver manualmente.
\begin{center}
\includegraphics[width=.9\linewidth]{figs/resolve_conflict_web.png}
\end{center}
\end{frame}

\begin{frame}[label={sec:org9b11ca0}]{Consejos}
\begin{itemize}
\item Las ramas accesorias utilizadas se pueden eliminar una vez finalizado el proceso.

\item Este proceso se debe repetir tantas veces como sea necesario para realizar cambios de forma colaborativa.

\item \alert{No olvides hacer \emph{pull} antes de iniciar una nueva interacción con el repositorio}.
\end{itemize}

\begin{block}{Más información en:}
\begin{itemize}
\item Página Web: \href{https://guides.github.com/introduction/flow/}{Understanding the GitHub Flow}
\item Vídeo: \href{https://youtu.be/PBI2Rz-ZOxU}{Understanding the GitHub Flow}
\end{itemize}
\end{block}
\end{frame}


\subsection{Tareas y tableros de discusión (\texttt{issues})}
\label{sec:org070e048}
\subsection{Herramientas gráficas para el análisis de un repositorio}
\label{sec:org891a9d4}
\begin{frame}[label={sec:org9039673}]{Insights}
Toda la actividad realizada en un repositorio puede verse de manera gráfica a través del botón \emph{Insights} en la web del repositorio en GitHub\footnote{Más detalles en \href{https://help.github.com/categories/visualizing-repository-data-with-graphs/}{Ver información del repositorio de forma gráfica}.}. Por ejemplo,

\begin{itemize}
\item Contribución de los integrantes del equipo
\item Estructuras de ramas de un repositorio
\item Histórico de cambios en un repositorio
\end{itemize}
\end{frame}

\begin{frame}[label={sec:org3ea72cf}]{Contribución de los integrantes del equipo}
\begin{center}
\includegraphics[width=.9\linewidth]{figs/repo_contributors_specific_graph.png}
\end{center}
\end{frame}

\begin{frame}[label={sec:org45e952d}]{Estructura de ramas de un repositorio}
\begin{center}
\includegraphics[width=.9\linewidth]{figs/repo_network_graph.png}
\end{center}
\end{frame}

\begin{frame}[label={sec:orga8ec7e6}]{Cambios en un repositorio}
\begin{center}
\includegraphics[width=.9\linewidth]{figs/repo_code_frequency_graph_dotcom.png}
\end{center}
\end{frame}

\section{GitHub Classroom}
\label{sec:org699be20}
\section{Publicación de páginas web en GitHub}
\label{sec:org379e77c}
\end{document}